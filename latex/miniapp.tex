\documentclass[aspectratio=43]{beamer}
% Theme works only with a 4:3 aspect ratio
\usetheme{CSCS}

\usepackage{tikz}
\usepackage{pgfplots}
\usepackage{pgfplotstable}
\usetikzlibrary{pgfplots.groupplots,spy,patterns}
\usepackage{listings}
\usepackage{color}
\usepackage{tcolorbox}
\usepackage{anyfontsize}
\usepackage{xspace}
\usepackage{graphicx}

% define footer text
\newcommand{\footlinetext}{CUDA MiniApp}

% Select the image for the title page
\newcommand{\picturetitle}{cscs_images/image5.pdf}

% fonts for maths
\usefonttheme{professionalfonts}
\usefonttheme{serif}

% source code listing
\newcommand{\axpy}{{\ttfamily axpy}\xspace}

% set indent to a more reasonable level (so that itemize can be used in columns)
\setlength{\leftmargini}{20pt}

%\lstset{
%    language=[ANSI]C++,
%    basicstyle=\lstinlinefont{blue!30!black},
%    breaklines=true
%}
\lstdefinestyle{terminal}{
    showstringspaces=false,
    backgroundcolor=\color{white},
    basicstyle=\lstfont{black},
    identifierstyle=\lstfont{black},
    keywordstyle=\lstfont{black},
    numberstyle=\lstfont{black},
    stringstyle=\lstfont{black},
    commentstyle=\lstfont{black},
    emph={
        aprun, make
    },
    emphstyle={\lstfont{black}},
    breaklines=true
}

\lstset{
    language=[ANSI]C++,
    showstringspaces=false,
    backgroundcolor=\color{white!10!black},
    basicstyle=\lstfont{white},
    identifierstyle=\lstfont{white},
    keywordstyle=\lstfont{magenta!40!white},
    numberstyle=\lstfont{white},
    stringstyle=\lstfont{cyan},
    commentstyle=\lstfont{yellow!30!white},
    moredelim=[is][\lstfont{green!50!white}]{@}{@},
    emph={
        cudaMalloc, cudaFree,
        cudaMallocHost, cudaFreeHost,
        cudaMemcpyAsync, cudaMemcpy, cudaMemcpyHostToDevice, cudaMemcpyDeviceToHost,
        cudaSuccess, cudaGetLastError, cudaGetErrorString,
        cudaErrorMemoryAllocation, cudaError_t,
        cudaOccupancyMaxPotentialBlockSize,
        __global__, __shared__, __device__, __host__,
        __syncthreads,
        cudaEvent_t, cudaStream_t,
        cudaEventCreate, cudaEventSynchronize, cudaEventDestroy,
        cudaEventElapsedTime, cudaEventQuery, cudaEventRecord,
        cudaStreamWaitEvent,
        threadIdx, blockIdx, blockDim, gridDim,
        cudaStream_t, cudaStreamCreate, cudaStreamDestroy,
        cudaMemcpyDeviceToDevice, cudaMemcpyHostToHost,
    },
    emphstyle={\lstfont{green!50!white}},
    breaklines=true
}

\definecolor{codenumber}{rgb}{0.5,0.5,0.5}
\definecolor{codekeyword}{rgb}{0.9,0.4,0.7}
\definecolor{codeCUDA}{rgb}{1.0,0.6,0.6}

\lstdefinestyle{boxcuda}{
    language=[ANSI]C++,
    showstringspaces=false,
    backgroundcolor=\color{white!10!black},
    basicstyle=\lstfont{white},
    identifierstyle=\lstfont{white},
    keywordstyle=\lstfont{magenta!40!white},
    numberstyle=\lstfont{white},
    stringstyle=\lstfont{cyan},
    commentstyle=\lstfont{yellow!30!white},
    moredelim=[is][\lstfont{green!50!white}]{@}{@},
    emph={
        cudaMalloc, cudaFree,
        cudaMallocHost, cudaFreeHost,
        cudaMemcpyAsync, cudaMemcpy, cudaMemcpyHostToDevice, cudaMemcpyDeviceToHost,
        cudaSuccess, cudaGetLastError, cudaGetErrorString,
        cudaErrorMemoryAllocation, cudaError_t,
        cudaOccupancyMaxPotentialBlockSize,
        __global__, __shared__, __device__, __host__,
        __syncthreads,
        cudaEvent_t, cudaStream_t,
        cudaEventCreate, cudaEventSynchronize, cudaEventDestroy,
        cudaEventElapsedTime, cudaEventQuery, cudaEventRecord,
        cudaStreamWaitEvent,
        threadIdx, blockIdx, blockDim, gridDim,
        cudaStream_t, cudaStreamCreate, cudaStreamDestroy,
        cudaMemcpyDeviceToDevice, cudaMemcpyHostToHost,
    },
    emphstyle={\lstfont{green!50!white}},
    breaklines=true
}

\lstdefinestyle{boxcudatiny}{
    language=[ANSI]C++,
    showstringspaces=false,
    backgroundcolor=\color{white!10!black},
    basicstyle=\lsttinyfont{white},
    identifierstyle=\lsttinyfont{white},
    keywordstyle=\lsttinyfont{magenta!40!white},
    numberstyle=\lsttinyfont{white},
    stringstyle=\lsttinyfont{cyan},
    commentstyle=\lsttinyfont{yellow!30!white},
    moredelim=[is][\lsttinyfont{green!50!white}]{@}{@},
    emph={
        cudaMalloc, cudaFree,
        cudaMallocHost, cudaFreeHost,
        cudaMemcpyAsync, cudaMemcpy, cudaMemcpyHostToDevice, cudaMemcpyDeviceToHost,
        cudaSuccess, cudaGetLastError, cudaGetErrorString,
        cudaErrorMemoryAllocation, cudaError_t,
        cudaOccupancyMaxPotentialBlockSize,
        __global__, __shared__, __device__, __host__,
        __syncthreads,
        cudaEvent_t, cudaStream_t,
        cudaEventCreate, cudaEventSynchronize, cudaEventDestroy,
        cudaEventElapsedTime, cudaEventQuery, cudaEventRecord,
        cudaStreamWaitEvent,
        threadIdx, blockIdx, blockDim, gridDim,
        cudaStream_t, cudaStreamCreate, cudaStreamDestroy,
        cudaMemcpyDeviceToDevice, cudaMemcpyHostToHost,
    },
    emphstyle={\lstfont{green!50!white}},
    breaklines=true
}

\DeclareTextFontCommand{\emph}{\bfseries\color{blue!70!black}}

% Please use the predifined colors:
% cscsred, cscsgrey, cscsgreen, cscsblue, cscsbrown, cscspurple, cscsyellow, cscsblack, cscswhite

\author{Sebastian Keller, Javier Otero, Prashanth Kanduri\\ and Ben Cumming, CSCS}
\title{CUDA: MiniApp}
\subtitle{}
\date{\today}

\begin{document}

% TITLE SLIDE
\cscstitle

%%%%%%%%%%%%%%%%%%%%%%%%%%%%%%%%%%%%%%%%%%%%
\begin{frame}[fragile]{Unit Tests}
%%%%%%%%%%%%%%%%%%%%%%%%%%%%%%%%%%%%%%%%%%%%
    Unit testing is an essential development practice:
    \begin{enumerate}
        \item Write tests for features before implementing them;
        \item Tests will pass when feature is finished;
        \item If a feature is broken in the future, the test will catch it immediately;
        \item Tests can also be used to catch performance regressions.
    \end{enumerate}
    The CUDA miniapp uses unit tests to validate the functions in \lst{linalg.cu}
    \begin{itemize}
        \item Initially all of the tests fail.
        \item The tests are run each time the project is built.
    \end{itemize}
    Use an open source unit testing framework for serious work, e.g. Google Test.
\end{frame}

%%%%%%%%%%%%%%%%%%%%%%%%%%%%%%%%%%%%%%%%%%%%
\begin{frame}[fragile]{Step 1: Host-Device Data Synchronization}
%%%%%%%%%%%%%%%%%%%%%%%%%%%%%%%%%%%%%%%%%%%%
    \begin{info}{The \lst{Field} class implements storage of the 2D fields}
        \begin{itemize}
            \item It is extended for CUDA to store a copy of each field in both host and device memory;
            \item The data is pointed to by the \lst{Field::host_pointer_} and \lst{Field::device_pointer_} members.
        \end{itemize}
    \end{info}
    The host and device copies need to be synchronised:
    \begin{enumerate}
        \item \emph{Implement} the member functions that synchronize the host and device data fields in \lst{data.h}.
        \item \emph{Update} the initial conditions on the device after computation in \lst{main.cu}.
    \end{enumerate}
    \begin{info}{}
        Two of the unit tests will pass when this task is finished.
    \end{info}
\end{frame}

%%%%%%%%%%%%%%%%%%%%%%%%%%%%%%%%%%%%%%%%%%%%
\begin{frame}[fragile]{Step 2: Linear Algebra}
%%%%%%%%%%%%%%%%%%%%%%%%%%%%%%%%%%%%%%%%%%%%
    Implement all of the BLAS level 1 linear algebra kernels in \lst{linalg.cu}:
    \begin{itemize}
        \item Look for the \lst{TODO}s;
        \item Each kernel is covered by a unit test;
        \item You will have to do some research into the cublas routines.
    \end{itemize}

    \begin{info}{}
        All of the unit tests will pass when this task is finished.
    \end{info}
\end{frame}

%%%%%%%%%%%%%%%%%%%%%%%%%%%%%%%%%%%%%%%%%%%%
\begin{frame}[fragile]{Step 3: Stencils}
%%%%%%%%%%%%%%%%%%%%%%%%%%%%%%%%%%%%%%%%%%%%
    Implement the stencils in \lst{operators.cu}
    \begin{itemize}
        \item Look for the TODOs.
    \end{itemize}
    The number of CG iterations and visualization can be used to validate results.
    \begin{itemize}
        \item \emph{Extra} can you use shared memory for the interior stencil?
        \item \emph{Extra} can you design an implementation that uses shared memory to implement the interior, boundary and corner stencils in one, simplified kernel?
    \end{itemize}
    How does time to solution compare with that for the serial version?
    \begin{itemize}
        \item Compare at different resolutions:\\128$\times$128, 512$\times$512, 1024$\times$1024.
    \end{itemize}
\end{frame}

\cscschapter{Thank you!}

\end{document}
