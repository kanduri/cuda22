\documentclass[aspectratio=43]{beamer}
% Theme works only with a 4:3 aspect ratio
\usetheme{CSCS}

\usepackage{tikz}
\usepackage{pgfplots}
\usepackage{pgfplotstable}
\usetikzlibrary{pgfplots.groupplots,spy,patterns}
\usepackage{listings}
\usepackage{color}
\usepackage{tcolorbox}
\usepackage{anyfontsize}
\usepackage{xspace}
\usepackage{graphicx}

% define footer text
\newcommand{\footlinetext}{Introduction to GPUs in HPC}

% Select the image for the title page
\newcommand{\picturetitle}{cscs_images/image5.pdf}

% fonts for maths
\usefonttheme{professionalfonts}
\usefonttheme{serif}

% source code listing
\newcommand{\axpy}{{\ttfamily axpy}\xspace}

% set indent to a more reasonable level (so that itemize can be used in columns)
\setlength{\leftmargini}{20pt}

\DeclareTextFontCommand{\emph}{\bfseries\color{blue!70!black}}

% Please use the predifined colors:
% cscsred, cscsgrey, cscsgreen, cscsblue, cscsbrown, cscspurple, cscsyellow, cscsblack, cscswhite

\author{Sebastian Keller, Javier Otero, Prashanth Kanduri\\ and Ben Cumming, CSCS}
\title{Advanced Features Overview}
\subtitle{}
% \date{\today}

\begin{document}

% TITLE SLIDE
\cscstitle

%++++++++++++++++++++++++++++++++
\cscschapter{Concurrency}
%++++++++++++++++++++++++++++++++

%%%%%%%%%%%%%%%%%%%%%%%%%%%%%%%%%%%%%%%%%%%%
\begin{frame}[fragile]{Concurrency}
%%%%%%%%%%%%%%%%%%%%%%%%%%%%%%%%%%%%%%%%%%%%
    \emph{Concurrency} is the ability to perform multiple CUDA operations simultaneously, including:
    \begin{itemize}
        \item CUDA kernels;
        \item Copying from host to device;
        \item Copying from device to host;
        \item Operations on the host CPU.
    \end{itemize}

    \begin{info}{What concurrency enables}
        \begin{itemize}
            \item Both CPU and GPU can work at the same time.
            \item Multiple tasks can run simultaneously on the GPU.
            \item communication and computation can be overlapped.
        \end{itemize}
    \end{info}

\end{frame}

%%%%%%%%%%%%%%%%%%%%%%%%%%%%%%%%%%%%%%%%%%%%
\begin{frame}[fragile]{The launch-execute sequence}
%%%%%%%%%%%%%%%%%%%%%%%%%%%%%%%%%%%%%%%%%%%%
    \begin{columns}[T]
        \begin{column}{0.5\textwidth}
            \begin{codecolumn}{Host code}
                \begin{lstlisting}[style=boxcudatiny]
kernel_1@<<<@...@>>>@(...);
kernel_2<@<<.@..@>>>@(...);
host_1(...);
host_2(...);
                \end{lstlisting}
            \end{codecolumn}
                The host (in order):
                \begin{itemize}
                    \item launch the kernels
                    \item execute host calls sequentially
                \end{itemize}
                The GPU:
                \begin{itemize}
                    \item executes asynchronously to host;
                    \item executes kernels sequentially.
                \end{itemize}
        \end{column}
        \begin{column}{0.5\textwidth}
            \includegraphics[width=\textwidth]{./images/async_null.pdf}
        \end{column}
    \end{columns}
\end{frame}

%%%%%%%%%%%%%%%%%%%%%%%%%%%%%%%%%%%%%%%%%%%%
\begin{frame}[fragile]{Overlapping Independent Operations}
%%%%%%%%%%%%%%%%%%%%%%%%%%%%%%%%%%%%%%%%%%%%
    The CUDA language and runtime libraries provide mechanisms for coordinating asynchronous GPU execution:

    \begin{itemize}
        \item Independent kernels and memory transfers can execute concurrently on different \emph{streams};
        \item \emph{CUDA events} can be used to synchronize streams and query the status of kernels and transfers.
    \end{itemize}

\end{frame}

%%%%%%%%%%%%%%%%%%%%%%%%%%%%%%%%%%%%%%%%%%%%
\begin{frame}[fragile]{Streams}
%%%%%%%%%%%%%%%%%%%%%%%%%%%%%%%%%%%%%%%%%%%%
    A CUDA stream is a sequence of operations that execute in \emph{issue order} on the GPU.
    \medskip

    \begin{info}{Streams and concurrency}
        \begin{itemize}
            \item Operations in different streams \emph{may} run concurrently
            \item Operations in the same stream \emph{are} executed sequentially
            \item If no stream is specified, all kernels are launched in the default stream
        \end{itemize}
    \end{info}

\end{frame}

%%%%%%%%%%%%%%%%%%%%%%%%%%%%%%%%%%%%%%%%%%%%
\begin{frame}[fragile]{Managing streams}
%%%%%%%%%%%%%%%%%%%%%%%%%%%%%%%%%%%%%%%%%%%%
    \begin{itemize}
        \item Streams can be created and destroyed:
        \begin{itemize}
            \item \lst{cudaStreamCreate(cudaStream_t* s)}
            \item \lst{cudaStreamDestroy(cudaStream_t s)}
        \end{itemize}
        \item Launch a kernel on a given stream: \\
            \begin{center} \lst{kernel``<<<``grid_dim, block_dim, shared_size, stream``>>>``(...)} \end{center}
        \item The default CUDA stream is the \lst{NULL} stream, or stream 0
    \end{itemize}

    \begin{code}{Basic cuda stream usage}
        \begin{lstlisting}[style=boxcudatiny]
// create stream
cudaStream_t stream;
cudaStreamCreate(&stream);
// launch kernel in stream
my_kernel@<<<@grid_dim, block_dim, shared_size, stream@>>>@(..)

...

// release stream when finished
cudaStreamDestroy(stream);
        \end{lstlisting}
\end{code}

\end{frame}

%%%%%%%%%%%%%%%%%%%%%%%%%%%%%%%%%%%%%%%%%%%%
\begin{frame}[fragile]{Concurrent Kernel Execution}
%%%%%%%%%%%%%%%%%%%%%%%%%%%%%%%%%%%%%%%%%%%%
    \begin{columns}[T]
        \begin{column}{0.45\textwidth}
            \begin{codecolumn}{Host code}
                \begin{lstlisting}[style=boxcudatiny]
kernel_1@<<<@_,_,_,stream_1@>>>@();
kernel_2@<<<@_,_,_,stream_2@>>>@();
kernel_3@<<<@_,_,_,stream_1@>>>@();
                \end{lstlisting}
            \end{codecolumn}
            \begin{itemize}
                \item \footnotesize \lst{kernel_1} and \lst{kernel_3} are serialized in \lst{stream_1}.
                \item \lst{kernel_2} can run asynchronously in \lst{stream_2}.
                \item \emph{Note} \lst{kernel_2} will only run concurrently if there are sufficient resources available on the GPU, i.e. if \lst{kernel_1} is not using all of the SMs.
            \end{itemize}
        \end{column}
        \begin{column}{0.6\textwidth}
            \includegraphics[width=\textwidth]{./images/async_two_streams.pdf}
        \end{column}
    \end{columns}
\end{frame}

%%%%%%%%%%%%%%%%%%%%%%%%%%%%%%%%%%%%%%%%%%%%
\begin{frame}[fragile]{Asynchronous copy}
%%%%%%%%%%%%%%%%%%%%%%%%%%%%%%%%%%%%%%%%%%%%

    \centering \lst{cudaMemcpyAsync(*dst, *src, size, kind, cudaStream_t stream = 0);}
    \begin{itemize}
        \item Takes an additional parameter stream, which is 0 by default.
        \item Returns immediately after initiating copy:
        \begin{itemize}
            \item Host can do work while copy is performed;
            \item Only if \emph{pinned memory} is used.
        \end{itemize}
        \item Copies in the same direction (i.e. H2D or D2H) are serialized.
        \begin{itemize}
            \item Copies from host$\rightarrow$device and device$\rightarrow$host are concurrent if in different streams.
        \end{itemize}
    \end{itemize}

\end{frame}

%%%%%%%%%%%%%%%%%%%%%%%%%%%%%%%%%%%%%%%%%%%%
\begin{frame}[fragile]{Pinned memory}
%%%%%%%%%%%%%%%%%%%%%%%%%%%%%%%%%%%%%%%%%%%%
    Pinned (or page-locked) memory will not be paged out to disk:
    \begin{itemize}
        \item The GPU can safely remotely read/write the memory directly without host involvement;
        \item Only use for transfers, because it easy to run out of memory.
    \end{itemize}

    \begin{info}{Managing pinned memory}
        \centering \lst{cudaMallocHost(**ptr, size);} and \lst{cudaFreeHost(*ptr);}
        \begin{itemize}
            \item Allocate and free pinned memory (\lst{size} is in bytes).
        \end{itemize}
    \end{info}

\end{frame}

%%%%%%%%%%%%%%%%%%%%%%%%%%%%%%%%%%%%%%%%%%%%
\begin{frame}[fragile]{}
%%%%%%%%%%%%%%%%%%%%%%%%%%%%%%%%%%%%%%%%%%%%
    \begin{info}{Asynchronous copy example: streaming workloads}
        Computations that can be performed independently, e.g. our \axpy example:
        \begin{itemize}
            \item Data in host memory has to be copied to the device, and the result copied back after the kernel is computed.
            \item Overlap copies with kernel calls by breaking the data into chunks.
        \end{itemize}
    \end{info}
    \includegraphics[width=\textwidth]{./images/overlap.pdf}
\end{frame}

%%%%%%%%%%%%%%%%%%%%%%%%%%%%%%%%%%%%%%%%%%%%
\begin{frame}[fragile]{CUDA events}
%%%%%%%%%%%%%%%%%%%%%%%%%%%%%%%%%%%%%%%%%%%%
    CUDA events can be used to coordinate operations on different GPU streams:
    \begin{itemize}
        \item Synchronize tasks in different streams, e.g.:
        \begin{itemize}
            \item Don't start work in stream \textit{a} until stream \textit{b} has finished;
            \item Wait until required data has finished copy from host before launching kernel.
        \end{itemize}
        \item Query status of concurrent tasks:
        \begin{itemize}
            \item Has kernel finished/started yet?
            \item How long did a kernel take to compute?
        \end{itemize}
    \end{itemize}
\end{frame}

%%%%%%%%%%%%%%%%%%%%%%%%%%%%%%%%%%%%%%%%%%%%
\begin{frame}[fragile]{Managing events}
%%%%%%%%%%%%%%%%%%%%%%%%%%%%%%%%%%%%%%%%%%%%
        \begin{itemize}
            \item Create and free \lst{cudaEvent_t}.
        \end{itemize}
    \centering\lst{cudaEventCreate(cudaEvent_t*);} and \lst{cudaEventDestroy(cudaEvent_t);}
        \begin{itemize}
            \item Enqueue an event in a stream.
        \end{itemize}
    \centering\lst{cudaEventRecord(cudaEvent_t, cudaStream_t);}
        \begin{itemize}
            \item Make host execution wait for event to occur.
        \end{itemize}
    \centering\lst{cudaEventSynchronize(cudaEvent_t);}
        \begin{itemize}
            \item Test if the work before an event in a queue has been completed.
        \end{itemize}
    \centering\lst{cudaEventQuery(cudaEvent_t)}
        \begin{itemize}
            \item Get time between two events.
        \end{itemize}
    \centering\lst{cudaEventElapsedTime(float*, cudaEvent_t, cudaEvent_t);}

\end{frame}

%%%%%%%%%%%%%%%%%%%%%%%%%%%%%%%%%%%%%%%%%%%%
\begin{frame}[fragile]{}
%%%%%%%%%%%%%%%%%%%%%%%%%%%%%%%%%%%%%%%%%%%%
    \begin{code}{Using events to time kernel execution}
        \begin{lstlisting}[style=boxcudatiny]
cudaEvent_t start, end;
cudaStream_t stream;
float time_taken;

// initialize the events and streams
cudaEventCreate(&start);
cudaEventCreate(&end);
cudaStreamCreate(&stream);

cudaEventRecord(start, stream); // enqueue start in stream
my_kernel@<<<@grid_dim, block_dim, 0, stream@>>>@();
cudaEventRecord(end, stream);   // enqueue end in stream
cudaEventSynchronize(end);      // wait for end to be reached
cudaEventElapsedTime(&time_taken, start, end);

std::cout << "kernel took " << 1000*time_taken << " s\n";

// free resources for events and streams
cudaEventDestroy(start);
cudaEventDestroy(end);
cudaStreamDestroy(stream);
        \end{lstlisting}
    \end{code}
\end{frame}

%%%%%%%%%%%%%%%%%%%%%%%%%%%%%%%%%%%%%%%%%%%%
\begin{frame}[fragile]{}
%%%%%%%%%%%%%%%%%%%%%%%%%%%%%%%%%%%%%%%%%%%%
    \begin{code}{Copy$\rightarrow$kernel synchronization}
        \begin{lstlisting}[style=boxcudatiny]
cudaEvent_t event;
cudaStream_t kernel_stream, h2d_stream;
size_t size = 100*sizeof(double);
double *dptr, *hptr;

// initialize
cudaEventCreate(&event);
cudaStreamCreate(&kernel_stream);
cudaStreamCreate(&h2d_stream);

cudaMalloc(&dptr, size);
cudaMallocHost(&hptr, size); // use pinned memory!

// start asynchronous copy in h2d_stream
cudaMemcpyAsync(dptr, hptr, size,
                cudaMemcpyHostToDevice, h2d_stream);
// enqueue event in stream
cudaEventRecord(event, h2d_stream);
// make kernel_stream wait for copy to finish
cudaStreamWaitEvent(kernel_stream, event, 0);
// enqueue my_kernel to start when event has finished
my_kernel@<<<@grid_dim, block_dim, 0, kernel_stream@>>>@();

// free resources for events and streams
cudaEventDestroy(event);
cudaStreamDestroy(h2d_stream);
cudaStreamDestroy(kernel_stream);
cudaFree(dptr);
cudaFreeHost(hptr);
        \end{lstlisting}
    \end{code}
\end{frame}

%%%%%%%%%%%%%%%%%%%%%%%%%%%%%%%%%%%%%%%%%%%%
\begin{frame}[fragile]{Exercises}
%%%%%%%%%%%%%%%%%%%%%%%%%%%%%%%%%%%%%%%%%%%%
    \begin{enumerate}
        \item Open \lstterm{include/util.hpp} and understand \lst{copy_to_``\{``host``/``device``\}``_async()} and \lst{malloc_pinned()}

        \item Open \lstterm{include/cuda_event.h} and \lstterm{include/cuda_stream.h}
        \begin{itemize}
            \item what is the purpose of these classes?
            \item what does \lst{cuda_stream::enqueue_event()} do?
        \end{itemize}

        \item Open \lstterm{async/memcopy1.cu} and run
        \begin{itemize}
            \item what does the benchmark test?
            \item what is the effect of turning on \lst{USE_PINNED}?\\Hint: try small and large values for \lst{n} (8, 16, 20, 24)
        \end{itemize}

        \item Inspect \lstterm{async/memcopy2.cu} and run
        \begin{itemize}
            \item what effect does changing the number of chunks have?
        \end{itemize}

        \item Inspect \lstterm{async/memcopy3.cu} and run
        \begin{itemize}
            \item how does it differ from \lstterm{memcopy2.cu}?
            \item what effect does changing the number of chunks have?
        \end{itemize}
    \end{enumerate}
\end{frame}


%%%%%%%%%%%%%%%%%%%%%%%%%%%%%%%%%%%%%%%%%%%%
\end{document}}
%%%%%%%%%%%%%%%%%%%%%%%%%%%%%%%%%%%%%%%%%%%%
